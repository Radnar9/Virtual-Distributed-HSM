%\pagestyle{empty}

% ----------------------------------------------------------------------
% P�gina do resumo em Portugu�s:
\selectlanguage{portuguese}
\vspace*{2cm}
\begin{center} \Large \bf Resumo
\end{center}
\vspace*{1cm} \setlength{\baselineskip}{0.6cm}

%Os documentos escritos em Portugu\^{e}s devem ter um resumo em Portugu\^{e}s e um resumo noutra l\'{i}ngua comunit\'{a}ria que contenham at\'{e} 300 palavras cada. Quando o conselho cient\'{i}fico autorizar a apresenta\c{c}\~{a}o do trabalho final escrito em l\'{i}ngua estrangeira, este deve ser acompanhado de um resumo adicional em Portugu\^{e}s de, pelo menos, 1200 palavras.



À medida que o cenário empresarial evolui continuamente, a necessidade de abordar os riscos de cibersegurança torna-se crítica para todos os tipos de organizações. Apesar dos investimentos substanciais por parte das grandes empresas, por outro lado, as mais pequenas muitas vezes não têm consciência destas ameaças ou não fizeram da proteção dos seus sistemas de informação uma prioridade máxima, deixando-os vulneráveis. O relatório de segurança da IBM de 2022 revela a consequência destas práticas, onde o custo médio global de brechas de dados atingiu um máximo histórico de \$4,35 milhões em 2022 (em comparação com \$4,24 milhões em 2021).

As abordagens tradicionais de segurança envolvem o uso de Módulos de Segurança de Hardware (HSMs), dispositivos criptográficos físicos dedicados que são especializados em salvaguardar a proteção de chaves criptográficas durante todo o seu ciclo de vida e realizar operações criptográficas fulcrais, destacando-se como as mais importantes, geração de chaves, assinaturas, e cifra/decifra. Estes dispositivos devem sempre ser confiáveis, possuindo, por isso, certificações reconhecidas internacionalmente que atestam as suas garantias de segurança.

Embora eficazes, os HSMs são dispendiosos e muitas vezes impraticáveis para empresas de menores dimensões. Desta forma, este trabalho propõe como solução um HSM virtual e distribuído, que permitirá com que startups e pequenas empresas desenvolvam estratégias de segurança em estágios mais iniciais, tendo ao seu dispor uma infraestrutura mais económica e prática em termos de implementação nos seus ambientes, sem que seja comprometida a segurança por não estarem a usar a versão física destes dispositivos. Isto é possível devido aos preços mais baixos quando se compara a hospedagem de um serviço distribuído com um serviço que precisa de replicar o seu sistema através de um conjunto de Módulos de Segurança de Hardware físicos e caros, a fim de garantir a disponibilidade da infraestrutura. A nossa abordagem permite que o sistem seja auto-hospedável e atua quase como uma solução de \textit{Software-as-a-Service} (SaaS), uma vez que está pronta a ser utilizada e é facilmente adaptável às necessidades e ao ambiente do cliente, exigindo menos esforço para colocá-la em prática.

O nosso HSM virtual aborda e melhora as limitações encontradas nas tentativas existentes de virtualizar HSMs, como SoftHSM, pmHSM e um HSM virtual apoiado por \textit{hardware}. O primeiro, utilizado apenas para fins de teste, executa as operações criptográficas localmente numa única máquina, não oferecendo garantias de segurança, já o segundo melhora algumas propriedades que faltavam no primeiro, incluindo segurança e disponibilidade, adaptando-o a uma solução distribuída para cumprir estes objetivos, enquanto que o terceiro propõe uma solução usando \textit{software} e \textit{hardware}, nomeadamente Intel SGX, um Ambiente de Execução Confiável (TEE) baseado em \textit{hardware}. Estas tentativas não atendem às expectativas e objetivos que pretendemos alcançar com este trabalho, uma vez que ao contrário das soluções anteriores, a nossa abordagem agrega as propriedades necessárias em uma só, nomeadamente disponibilidade, integridade e confidencialidade sem dependência de TEEs, mas em vez disso, contamos com um sistema distribuído de forma a atingir níveis de segurança similares aos que existem num dispositivo físico, permitindo ao nosso HSM ser tolerante a assincronias e faltas/intrusões, o que nenhuma das referidas tentativas teve em consideração. Mais especificamente, utilizamos um sistema de replicação de máquinas de estado tolerante a faltas bizantinas (BFT SMR), fornecido pelo BFT-SMaRt, uma vez que corresponde ao estado-de-arte para implementar este tipo de sistema de forma realista e prática.

Portanto, para atingir o nosso objetivo, estudámos protocolos de estado-de-arte eficientes na área de criptografia de \textit{threshold}, especificamente para as operações de geração de chaves distribuída, assinaturas distribuídas, e cifra simétrica distribuída, já que estas são as versões distribuídas das funcionalidades mais importantes de um HSM. Estes protocolos de \textit{threshold} correspondem a algoritmos criptográficos onde múltiplos servidores são necessários para realizar operações criptográficas, ao invés de se depender de um único dispositivo confiável. Esta alternativa exige que um determinado número de dispositivos esteja comprometido para que um adversário consiga violar a segurança do sistema. Com base neste estudo, desenvolvemos um HSM Virtual e Distribuído eficiente e robusto para fornecer as garantias de segurança referidas anteriormente.

O nosso projeto foi desenvolvido sobre o COBRA, uma \textit{framework} de protocolos  baseado em \textit{secret sharing} proativo e dinâmico que permite implementar confidencialidade em sistemas práticos de BFT SMR. Esta \textit{framework} permitiu-nos assegurar a proteção de chaves criptográficas geradas, através do seu protocolo de geração de polinómios distribuído, e adaptar este mesmo protocolo para implementar o algoritmo de geração de chaves distribuídas para um número arbitrário de diferentes curvas elípticas. Este protocolo é baseado em \textit{secret sharing}, um dos ramos da criptografia de \textit{threshold}. Este esquema protege a confidencialidade de um segredo armazenado, dividindo-o em \textit{n} fragmentos (ou \textit{shares}), onde uma porção destas pode ser, posteriormente, combinada para recuperar o segredo inicial. Contudo, combinar um número inferior ao necessário de \textit{shares}, não irá revelar nenhuma informação sobre o segredo. Além de possuir estas características, o esquema apresenta-se também como proativo e dinâmico, permitindo não só alterações no conjunto de entidades que possuem cada \textit{share}, como também possibilitando a renovação e verificabilidade de \textit{shares}, garantindo a sua integridade. O sistema BFT SMR e os seus benefícios como tolerância a faltas e assincronia, recuperação de faltas, e reconfigurações de grupo são fornecidos pelo BFT-SMaRt, sobre o qual o COBRA foi construído.

Em relação à implementação das assinaturas e da cifra simétrica de \textit{threshold}, ambos os protocolos são semelhantes quando observados os algoritmos ao alto nível. Um grupo de servidores calcula coletivamente uma assinatura ou resultado parcial sem expor qualquer informação sobre a chave privada e, posteriormente, uma entidade confiável, no nosso caso o cliente, irá receber esses resultados e terá a responsabilidade de agregá-los na assinatura ou cifra/decifra final.

As assinaturas escolhidas, como também os algoritmos de geração de chaves, foram Schnorr e BLS, uma vez que estes são muito mais fáceis de implementar na sua versão de \textit{threshold} do que outros, e são suportados por duas das blockchains mais significativas, Bitcoin e Ethereum, respetivamente. Relativamente à cifra e decifra, escolhemos o protocolo mais reconhecido, proposto no artigo \textit{Distributed Symmetric-key Encryption} (DiSE), que consiste numa construção genérica de cifra de \textit{threshold} autenticada baseada em qualquer função pseudoaleatória distribuída (DPRF). O DPRF é o componente mais importante do algoritmo e sua responsabilidade é gerar resultados parciais de forma determinística. Ser determinístico é o fator chave para possibilitar uma posterior decifra de um texto previamente cifrado por este mesmo protocolo.

Embora tenhamos enfatizado até aqui as pequenas empresas e o uso de um HSM, o nosso sistema poderá também ser usado como uma carteira de criptomoedas. Embora não esteja integrado diretamente numa blockchain e não possua algumas funcionalidades básicas, como a visualização do saldo da conta ou das transações efetuadas, este suporta as mais importantes, nomeadamente, a geração de chaves, que é feita no momento da criação da conta; o armazenamento seguro da chave privada, através da distribuição das suas \textit{shares} pelos servidores disponíveis; e também a assinatura de transações, uma vez que estão implementados algoritmos compatíveis com as principais blockchains.

Os resultados da avaliação experimental em termos de desempenho do nosso sistema mostram que as implementações que usam Schnorr (tanto para geração de chaves como para assinaturas) são pelo menos quatro vezes mais rápidas e escalam muito melhor quando comparadas com BLS e, no que diz respeito ao protocolo de cifra, à medida que o número de réplicas usadas aumenta, o seu desempenho reduz para cerca de metade. No entanto, a avaliação revela resultados promissores que certamente poderão ser ainda melhorados com otimizações futuras.

Após a conclusão do trabalho desenvolvido, para além desta dissertação, este projeto culminou ainda no desenvolvimento de um artigo científico para a conferência portuguesa INForum 2024, encaixando-se sobre o tema "Segurança de Sistemas de Computadores e Comunicações".

%Para documentos em Portugu�s: Resumo em portugu�s at� \textbf{300} palavras.
%Para documentos em l�ngua estrangeira: Resumo em portugu�s com pelo menos \textbf{1200} palavras.

\vfill

\begin{flushleft}
\textbf{Palavras-chave:}
Módulo de Segurança de Hardware, Carteira de Criptomoedas, Geração de Chaves Distribuída, Assinatura Distribuída, Cifra Simétrica Distribuída.
\end{flushleft}

\LIMPA
% Fim da p�gina do resumo em Portugu�s
% ----------------------------------------------------------------------
